\documentclass[11pt, a4paper]{article}
\usepackage[utf8]{inputenc}
\usepackage[spanish]{babel}
\usepackage[margin=1in]{geometry}
\usepackage{graphicx}

\renewcommand{\baselinestretch}{1.2}

\title{Memoria Final: K-meeting-on-PDE (OC2025-026)}
\author{}
\date{}

\begin{document}
\maketitle
\tableofcontents

\begin{abstract}
Memoria final de la actividad «K-Meeting on PDE: Analysis, Numerical Methods and Applications», organizada dentro del Plan propio de estímulo a la Investigación y Transferenia del Vicerrectorado de Investigación y Transferencia de la Universidad de Cádiz (ayuda OC2025-026).
\end{abstract}

\section{Titulo}
K-Meeting on PDE: Analysis, Numerical Methods and Applications
\section{Lugar y fechas de celebración}
Cádiz, Edificio Constitución 1812 (antiguo Aulario La Bomba), del 11/09/25 al 12/09/25
\section{Motivación y Objetivos}
El desarrollo de las Ecuaciones en Derivadas Parciales (PDEs en inglés) y métodos numéricos asociados es el principal del el grupo de investigación FQM-315 «Análisis teórico y numérico de modelos de las ciencias experimentales» dentro del Departamento de Matemáticas de la Universidad de Cádiz. Muchos son los contactos que sus miembros han desarrollado a lo largo de estos años y varios son los doctorandos se han formado o están en formación.

El esfuerzo de los miembros del grupo FQM-315 ha propiciado colaboraciones con prestigiosos investigadores españoles y de países como  Italia, Alemania, Framcia. USA, Chile, Marruecos o Argelia, con los que ha mantenido un flujo activo de intercambio de
conocimientos. Uno de los colaboradores más estrechos es el profesor Francisco Guillén
González, que ha sido director o codirector de tesis de distintos integrantes de el grupo, con los que ha publicado decenas de artículos. 

En este encuentro, y coincidiendo con el 60 cumpleaños del profesor Francisco Guillén González, perseguía reunir a algunos de los matemáticos internacionales de primer nivel que  han colaborado con él a lo largo de los años, con la finalidad de fortalecer las estrechas relaciones académicas y personales que han venido tejiendo con los años en torno a la actividad investigadora del profesor Guillén González.

El encuentro se centró no solamente en el análisis teórico y numérico de EDP, sino en su aplicación a modelos matemáticos aplicados a otras áreas de las ciencias. En este sentido, se contó con la participación de neurocientíficos de las universidades de Sevilla y Cádiz con quienes se ha venido colaborando estrechamente en los últimos años y con quienes existe un amplio campo de colaboración en el futuro.

\begin{figure}
\caption{Horario}
\begin{center}
\includegraphics[width=0.79\textwidth]{K-schedule.png}
\end{center}
\label{fig:horario}
\end{figure}

\section{Programa científico}
En el programa del encuentro (figura~\ref{fig:horario}) se contemplaron 8 conferencias plenarias, 2 de las cuales fueron realizadas por parte de investigadores de universidades extranjeras.
Las ponencias versaron en torno a temas candentes de investigación en torno a la temática del congreso y fueron impartidas por colaboradores directos del profesor Guillén González (que lideró dos mesas redondas). Estas ponencias fueron introducidas en una conferencia inaugural por una de sus primeras colaboradoras.

\medskip
\noindent
Ponencias plenarias:
\begin{enumerate}
\item \emph{Giordano Tierra Chica} (Norh Texas University, EEUU). 
\item \emph{Carmen Castro González} (Universidad de Cádiz). 
\item \emph{J. Rafael Rodríguez Galván} (Universidad de Cádiz).
\item \emph{Noelia Ortega Román} (Universidad de Cádiz).
\item \emph{Marko Rojas Medar} (Universidad de Tarapacá, Chile). 
\item \emph{María Ángeles Rodriguez Bellido} (Universidad de Sevilla). 
\item \emph{Juan Vicente Gutiérrez Santacreu} (Universidad de Sevilla). 
\item \emph{Daniel Acosta Soba} (Universidad de Cádiz). 
\end{enumerate}

\medskip
\noindent
Introducción (ponencia innaugural):
\begin{itemize}
\item \emph{Mª Victoria Redondo Neble} (Universidad de Cádiz). 
\end{itemize}

\medskip
\noindent
Coordinación de debate en mesas redondas:
\begin{itemize}
\item \emph{Francisco Guillén González} (Universidad de Cádiz). 
\end{itemize}

\section{Memoria Económica}
Los gastos del congreso han sido totalmente cubiertos utilizando la \emph{financiación} de \textbf{188.50€} concedida por el Plan Propio de estímulo a la Investigación y Transferencia de la Universidad de Cádiz (ayuda a organización de congresos OC2025-26).
Estos gastos, detallados en la solicitud de la ayuda OC2025-026, están en trámite y serán cubiertos por la orgánica (20DPMACO13) habilitada para esta ayuda:


\begin{itemize}
  \item \emph{Manutención}: dos almuerzos del profesor Francisco Guillén González (días 11 y 12 de septiembre).
  \item \emph{Alojamiento}: Estancia del profesor Francisco Guillén González en el Colegio Mayor del 11 al 12 de septiembre.
  \item Gastos de \emph{desplazamiento} del profesor Francisco Guillén González.
\end{itemize}

\section{Comité Organizador}
  \begin{itemize}
  \item \emph{Daniel Acosta Soba} (Universidad de Cádiz). 
  \item \emph{Mª Victoria Redondo Neble} (Universidad de Cádiz).
  \item \emph{J. Rafael Rodríguez Galván} (Universidad de Cádiz).
\end{itemize}

\section{Participantes}
% Vicky, Rafa, Dani, Juanvi, MA, Giordano, Marko, Kisko, Carmen, Pedro, Aurora,  
En el encuentro participaron 18 personas, 5 de las cuales provenientes de Universidades en paíseis extranjeros: Norh Texas University (EEUU), Universidad de Tarapacá (Chile), Universidad de Cagliari (Italia). Dos personas participaron de  manera no presencial.

\end{document}
